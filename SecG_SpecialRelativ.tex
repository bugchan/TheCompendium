
\section{Special Relativity}
(such as introductory concepts, time dilation,
length contraction, simultaneity,
energy and momentum, four-vectors
and Lorentz transformation,
velocity addition)

\subsection{Introductory concepts} 
\center
\begin{tabular}{|c|c|}
\hline

 An \textit{event} is ... & ... a location and a time: $(x, y, z, c t).$

\\ \hline

$\beta =$ & $v/c$ generally $<1$

\\ \hline

$\gamma = $ & $\dfrac{1}{\sqrt{1 - \beta^2}}$ generally $>1$

\\ \hline
\end{tabular}
\flushleft

%%%%

\subsection{Lorentz invariant intervals}
\center
\begin{tabular}{|c|c|}
\hline

Lorentz invariant interval & $ s^2 = \Delta x^2 + \Delta y^2 + \Delta z^2 - (c \Delta t)^2 $\\
& $ = \Delta x'^2 + \Delta y'^2 + \Delta z'^2 - (c \Delta t')^2$

\\ \hline

Time-like interval,  & Enough time passes between two events that \\ 
description and inequality:  &they could be causally related. \\
& $s^2 < 0$ or $c^2 \Delta t^2 > \Delta r^2$

\\ \hline

Proper time interval & Would be measured by an observer traveling\\
$\Delta \tau $ & between two time-like events in an inertial frame.  \\
 & $ \Delta \tau = \sqrt{\Delta t^2 - \dfrac{ \Delta r^2}{c^2}} $

\\ \hline

Light-like interval & Events which which occur to or are \\
 & initiated by a photon along its path. \\
 & $s^2 = 0 $ or $c^2 \Delta t^2 = \Delta r^2$

\\ \hline

\end{tabular}

\begin{tabular}{|c|c|}
\hline

Space-like interval & Not enough time between the events for \\
 & the possibility of a causal relation. There exists a \\
  & ref. frame in which the two events are simultaneous, \\ 
   & but no ref. frame in which they occur in the spatial location. \\ 
   & $s^2 > 0$ or $c^2 \Delta t^2 < \Delta r^2$
   
\\ \hline

Proper distance & The measurement of space-like separation between events. \\
 $\Delta \sigma $ & $\Delta \sigma = \sqrt{s^2} = \sqrt{\Delta r^2 - c^2 t^2} $

\\ \hline
\end{tabular}
\flushleft



\subsection{Time dilation} 
\center
\begin{tabular}{|c|c|}
\hline

`Moving clocks run ... & ... slower.'

\\ \hline

Proper time & Time read on the face of the moving clock (i.e. clock's ref. fr.) \\
$\Delta \tau = $ & $\dfrac{\Delta t}{\gamma}$ \\ 
 & Lorentz invariant
 
\\ \hline
\end{tabular}
\flushleft

%%%%

\Table{
\hline

Relativistic Doppler Effect

&

\MiniPg{.6}{\center
$\dfrac{\lambda_o}{\lambda_s} = \sqrt{\dfrac{1 + \beta}{1 - \beta}}$

$\beta = \dfrac{ (\frac{\lambda_o}{\lambda_s})^2 - 1}{(\frac{\lambda_o}{\lambda_s})^2 + 1}$

Where $\beta$ is $v/c$ and $+v$ means the object is going away from the observer.
}

\\ \hline
}

%%%%%%%%%%%%%%%%%%%%%%%%%%%%%%%%%%%%

\subsection{Length contraction} 
\center
\begin{tabular}{|c|c|}
\hline

`Moving sticks are...  & ... shorter.'

\\ \hline

Proper length & Length measured in ref. fr. of the stick. \\
$\Delta L_0$ & $\Delta L_0 = \gamma \Delta L$
 
 \\ \hline
\end{tabular}
\flushleft

%%%%%%%%%%%%%%%%%%%%%%%%%%

\subsection{Simultaneity}

%%%%%%%%%%%%%%%%%%%%%%%%%%

\subsection{Energy and momentum} 
\Table{
\hline

\MiniPg{.4}{
\center

For a particle with rest mass $m_0$ and velocity $\vec v$

$\bold{p} = $

}

& 

$\gamma m_0 \bold{v}$

\\ \hline

$E = $ & $\gamma m_0 c^2 = RestE + KE$

\\ \hline

Kinetic energy & $KE = E - RestE = mc^2 - m_0 c^2 = m_0 c^2 (\gamma - 1)$

\\ \hline

Energy and momentum relation

& 

\MiniPg{.6}{
\center
\MPalign{
p^2 c^2 & = \gamma^2 m_0^2 v^2 c^2 \\
		& = \gamma^2 m_0^2 \dfrac{v^2}{c^2} c^4 \\
		& = \gamma^2 \BigP{ m_0^2 \dfrac{v^2}{c^2} c^4 - m_0^2 c^4 + m_0^2 c^4} \\
		& = \gamma^2 \BigP{ m_0^2 c^4 \BigP{ \dfrac{v^2}{c^2} - 1 } + m_0^2c^4} \\
		& = -m_0^2c^4 + \gamma^2 m_0^2 c^4
 }
 
Rearrange: $\boxed{E^2 = (pc)^2 + (m_0 c^2)^2}$

} 
 \\ \hline
}

%%%%%%%%%%%%%%%%%%%%%%%%%%%%%%%%

\subsection{Lorentz transformation} 
\center
\begin{tabular}{|c|c|}
\hline
Lorentz transformation & \\

$x' = $ & $\gamma(x - \beta ct)$ \\

$ct' = $ & $ \gamma(ct - \beta x)$

\\ \hline

Inverse Lorentz Transformation & $v \rightarrow -v$ \\

$x = $ & $\gamma(x'+ \beta ct') $ \\

$ct = $ & $\gamma (ct' + \beta x')$

\\ \hline

Electromagnetic fields & $\bold{E'}_{\parallel} = \bold{E}_{\parallel}$ \\
parallel and perpendicular to $\bold{v}$ & $\bold{B'}_{\parallel} = \bold{B}_{\parallel}$ \\
& $\bold{E'_{\bot}} = \gamma (\bold{E}_{\bot} + \bold{v} \times \bold{B})$ \\
& $\bold{B'}_{\bot} = \gamma\Big(\bold{B}_{\bot} - \dfrac{1}{c^2}\bold{v}\times\bold{E}\Big)$

\\ \hline
\end{tabular}
\flushleft

\subsection{Derivation of $E=mc^2$ using four-vectors} 
\center
\begin{tabular}{|c|c|}
\hline

$x^\mu =  $& $(x,y,z,ct) $ where $\mu = 1,2,3,4$

\\ \hline

$x_\mu = $ & $(x,y,z,-ct)$

\\ \hline

$x^\mu x_\mu = $ & $(x^2+y^2+z^2-(ct)^2$

\\ \hline

$\Delta x^\mu =$ & $(\Delta x, \Delta y, \Delta z, c \Delta t) = ( \Delta \vec r, c \Delta t)$

\\ \hline

$\dfrac{\Delta x^\mu}{\Delta \tau} = $ & $\gamma \dfrac{\Delta x^\mu}{\Delta t} $\\

 & $ = (\gamma v_x, \gamma v_y, \gamma v_z, \gamma c) $ \\

four-velocity & $ = (\gamma \vec v, \gamma c) $

\\ \hline

$m \dfrac{\Delta x^\mu}{\Delta \tau} = $ &  $m \gamma \dfrac{\Delta x^\mu}{\Delta t} $\\

four-momentum & $=  (\gamma m \vec v, \gamma m c) = p^\mu$

\\ \hline

$p^\mu p_\mu = $ & rest frame,  $\gamma = 1$, so $p^\mu = (0, mc)$ \\

rest and non-rest frames & non-rest, $p^\mu = (\gamma m \vec v, \gamma m c) $ \\

 &  four-vector dot product is invariant, \\ 
 & so $p^\mu p_\mu$ is the same for rest and non-rest: \\
& $p^\mu p_\mu = (\gamma m v)^2 - (\gamma m c)^2 = -(mc)^2$

\\ \hline

\end{tabular}

\begin{tabular}{|c|c|}
\hline

 $(\gamma m v)^2 - (\gamma m c)^2 = -(mc)^2$& $(p)^2 - (E/c)^2 = -(mc)^2$ \\
 Rewrite, solve in terms of $E$. & Multiply by $c^2$ and rearrange. \\ 
 & $E^2 = (pc)^2 + (mc^2)^2$

\\ \hline
\end{tabular}
\flushleft


\subsection{Four-vectors and Lorentz transformation} 


\subsection{Relativistic Velocity Addition} 
\center
\begin{tabular}{|c|c|}
\hline

$v_{tot} = $ & $ \dfrac{v_1 + v_2}{1 + \dfrac{v_1 v_2}{c^2}} $

 \\ \hline
\end{tabular}
\flushleft


\subsection{Twin astronauts problem} 



